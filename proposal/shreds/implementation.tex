\subsection{Implementation}
\label{shreds:sec:impl}

We built S-compiler based on LLVM~\cite{lattner2004llvm} and its C front-end  Clang~\cite{clang}. We built S-driver with Linux as the reference OS. The implemented system was deployed and evaluated on a quad-core ARM Cortex-A7 computer (Raspberry Pi 2 Model B running Linux 4.1.15).

\point{S-compiler}
The modular and pass-based architecture of LLVM allows us to take advantage of the existing analyzers and easily extends the compilation pipeline. 
S-compiler adds two new passes to LLVM: the shred analysis pass and the security instrumentation pass. Both operate on LLVM bitcode as the IR. 

The analysis pass carries out the checks on the usages and security properties of shreds, as described in \S~\ref{shreds:sec:design}. 
We did not use LLVM's built-in data flow analysis for those checks due to its overly heuristic point-to analysis and the unnecessarily conservative transfer functions.
Instead, we implemented our specialized data flow analysis based on the basic round-robin iterative algorithm, with weak context sensitivity and a straightforward propagation model (i.e., only tracking value-conserving propagators).
We also had to extend LLVM's compilation pipeline because it by default only supports intra-module passes while S-compiler needs to perform inter-module analysis. We employed a linker plugin, called the Link-Time Optimization (LTO), to cross link the IR of all compilation modules and feed the linked IR to our analyzers. 


The instrumentation pass uses the LLVM IR Builder interfaces to insert security checks into the analyzed IR, which are necessary for enforcing the in-shred control flow regulations and preventing dynamic data leaks. 

\IncMargin{1em}
\begin{algorithm}
\small
\SetKwData{Spool}{s\_pool}
\SetKwData{Owner}{s\_owner}
\SetKwData{FaultThread}{fault\_thread}
\SetKwData{CPUDomain}{cpu\_domain}
\SetKwData{SpoolDomain}{s\_pool\_domain}
\SetKwFunction{FindSpool}{FindSpool}
\SetKwFunction{GetOwner}{GetOwner}
\SetKwFunction{GetCPUDomain}{GetCPUDomain}
\SetKwFunction{GetSpoolDomain}{GetSpoolDomain}
\SetKwFunction{BadArea}{bad\_area}
\SetKwFunction{GoodArea}{good\_area}
\SetKwFunction{RestoreDACR}{RestoreDACR}
\SetKwFunction{UnlockSPool}{UnlockSPool}
\SetKwFunction{AdjustSPool}{AdjustSPool}
\SetKwFunction{LockActiveSPoolList}{LockOtherActiveSPools}
\SetKwInOut{Input}{input}
\SetKwInOut{Output}{result}
\Input{The faulting virtual address $fault\_addr$}
\Output{Recover from the domain fault, or kill the faulting thread}
\BlankLine
\emph{/*Identity check*/}\\
  \Spool$\leftarrow$ \FindSpool{$fault\_addr$}\;
  \Owner$\leftarrow$ \GetOwner{$\Spool$}\;
  \If{\FaultThread is NOT in shred}{goto \BadArea}
  \If{\FaultThread is NOT \Owner}{goto \BadArea}
\BlankLine
\emph{/*Recover from domain fault*/}\\
  \CPUDomain$\leftarrow$ \GetCPUDomain{}\;
  \SpoolDomain$\leftarrow$ \GetSpoolDomain{\Spool}\;
   \If{\Spool is unlocked}
   {
   		  
          \If{\CPUDomain $=$ \SpoolDomain}
          {
            \emph{/*No need to change domain for s\_pool*/} \\
            \RestoreDACR{}\;
          }
          \Else
   		  {
   		    \AdjustSPool{\CPUDomain}  
   		  }
   }
    \Else
    {
    	\UnlockSPool{\CPUDomain} 
    }
   \LockActiveSPoolList{\Spool}\;
\caption{Domain Fault Handler}\label{dom_fault_handler}
\end{algorithm}
\DecMargin{1em}

\point{S-driver}
We built S-driver into a Loadable Kernel Module (LKM) for Linux. 
S-driver creates a virtual device file ({\tt /dev/shreds}) to handle the {\tt ioctl} requests made internally by the shred APIs. 
It uses 13 out of 16 memory domains to protect s-pools because the recent versions of Linux kernel for ARM already occupies 3 domains (for isolating device, kernel, and user-space memory). 
S-driver uses the available domains to protect unlimited s-pools and controls each CPU's access to the domains as described in \S~\ref{shreds:sec:design}.
Since Linux does not provide callback interfaces for drivers to react to scheduling events, in order to safely handle context switches or signal dispatches in shreds, 
S-driver dynamically patches the OS scheduler so that, during every context switch, the DACR of the current CPU is reset, which locks the open s-pool, if any. 
The overhead of this operation is negligible because resetting the DACR only takes a single lightweight instruction. 
To capture illegal access to s-pools and lazily adjust domain assignments, 
S-driver registers itself to be the only handler of domain faults and is triggered whenever a domain violation happens. 
Algorithm~\ref{dom_fault_handler} shows how S-driver handles a domain fault. 
Purely implementing S-driver as a LKM allows shreds to be introduced into a host without installing a custom-build kernel image.
