\subsection{Methodology}
\label{rpo:sec:methodology}

Our methodology consists of three main phases. We seed our system with pages
from the Common Crawl archive to extract \textit{candidate} pages that include
at least one stylesheet using a relative path. To determine whether these
candidate pages are \textit{vulnerable}, we attempt to inject style directives
by requesting variations of each page's URL to cause \textit{path confusion} and
test whether the generated response reflects the injected style directives.
Finally, we test how often vulnerable pages can be \textit{exploited} by
checking whether the reflected style directives are parsed and used for
rendering in a web browser.

\subsubsection{Candidate Identification}
\label{rpo:sec:methodology:candidate}

For finding the initial seed set of candidate pages with relative-path
stylesheets, we leverage the Common Crawl from August 2016, which contains more
than 1.6 billion pages. We use a Java HTML parser to filter any pages containing
only inline CSS or stylesheets referenced by absolute URLs, leaving us with over
203 million pages on nearly 6 million sites. For scalability purposes, we
further reduce the set of candidate pages in two steps:

\begin{enumerate}

\item We retain only pages from sites listed in the Alexa Top 1 million ranking,
which reduces the number of candidate pages to 141 million pages on 223 thousand
sites.

\item We observed that many sites use templates customized through query strings
or path parameters. We expect these templates to cause similar vulnerability and
exploitability behavior for their instantiations, thus we can speed up our
detection by grouping URLs using the same template, and testing only one random
representative of each group.

\end{enumerate}

Since our methodology contains a step during which we actively test whether a
vulnerability can be exploited, we remove from the candidate set all pages
hosted on sites in \texttt{.gov}, \texttt{.mil}, \texttt{.army}, \texttt{.navy},
and \texttt{.airforce}. The final candidate set consists of 137 million pages
(31 million page groups) on 222 thousand sites.

\subsubsection{Vulnerability Detection}
\label{rpo:sec:methodology:vulnerable}

To determine whether a candidate page is vulnerable, we implemented a
lightweight crawler based on the Python Requests API. At a high level, the
crawler simulates how a browser expands relative paths and tests whether style
directives can be injected into the resources loaded as stylesheets using path
confusion. For each page group from the candidate set, the crawler randomly
selects one representative URL and mutates it according to a number of
techniques explained below. The style directive we inject to test for reflection
vulnerabilities is shown in the legend of Figure~\ref{rpo:fig:taint_techniques}.

\begin{figure}
    \centering

    \begin{subfigure}[t]{1\linewidth}
        \begin{lstlisting}
        http://victim.com/page.asp
        http://victim.com/page.asp@@/@@**PAYLOAD**@@//@@
        http://victim.com/page.asp@@/@@**PAYLOAD**@@/@@style.css
        \end{lstlisting}
        \caption{Path Parameter (Simple)}
        \label{rpo:fig:taint:parameter_simple}
    \end{subfigure}

    \begin{subfigure}[t]{1\linewidth}
        \begin{lstlisting}
        http://victim.com/page.php/param1/param2
        http://victim.com/page.php/**PAYLOAD**param1/**PAYLOAD**param2@@//@@
        http://victim.com/page.php/**PAYLOAD**param1/**PAYLOAD**param2@@/@@style.css
        \end{lstlisting}
        \caption{Path Parameter (PHP or ASP)}
        \label{rpo:fig:taint:parameter_php}
    \end{subfigure}

    \begin{subfigure}[t]{1\linewidth}
        \begin{lstlisting}
        http://victim.com/page.jsp;param1;param2
        http://victim.com/page.jsp;**PAYLOAD**param1;**PAYLOAD**param2@@//@@
        http://victim.com/page.jsp;**PAYLOAD**param1;**PAYLOAD**param2@@/@@style.css
        \end{lstlisting}
        \caption{Path Parameter (JSP)}
        \label{rpo:fig:taint:parameter_jsp}
    \end{subfigure}

    \begin{subfigure}[t]{1\linewidth}
        \begin{lstlisting}
        http://victim.com/dir/page.aspx
        http://victim.com/**PAYLOAD**@@/..%2F@@dir/**PAYLOAD**@@/..%2F@@page.aspx@@//@@
        http://victim.com/**PAYLOAD**@@/..%2F@@dir/**PAYLOAD**@@/..%2F@@page.aspx@@/@@style.css
        \end{lstlisting}
        \caption{Encoded Path}
        \label{rpo:fig:taint:path}
    \end{subfigure}

    \begin{subfigure}[t]{1\linewidth}
        \begin{lstlisting}
        http://victim.com/page.html?k1=v1&k2=v2
        http://victim.com/page.html@@%3F@@k1=**PAYLOAD**v1&k2=**PAYLOAD**v2@@//@@
        http://victim.com/page.html@@%3F@@k1=**PAYLOAD**v1&k2=**PAYLOAD**v2@@/@@style.css
        \end{lstlisting}
        \caption{Encoded Query}
        \label{rpo:fig:taint:query}
    \end{subfigure}

    \begin{subfigure}[t]{1\linewidth}
        \begin{lstlisting}
        http://victim.com/page.php?key=value
        http://victim.com/page.php@@//@@?key=value
        http://victim.com/page.php@@/@@style.css

        Original Cookie: k1=v1; k2=v2
         Crafted Cookie: k1=**PAYLOAD**v1; k2=**PAYLOAD**v2
        \end{lstlisting}
        \caption{Cookie}
        \label{rpo:fig:taint:cookie}
    \end{subfigure}

    \caption{Various techniques of \redbold{path confusion} and \bluebold{style
    injection}. In each example, the first URL corresponds to the regular page, and
    the second one to the page URL crafted by the attacker. Each HTML page is
    assumed to reference a stylesheet at \protect\texttt{../style.css}, resulting in the
    browser expanding the stylesheet path as shown in the third URL.
    \bluebold{PAYLOAD} corresponds to
    \bluebold{\texttt{\%0A\{\}body\{background:NONCE\}}} (simplified), where
    \texttt{NONCE} is a randomly generated string.}

    \label{rpo:fig:taint_techniques}
\end{figure}


\paragraph{Path Parameter}

A number of web frameworks such as PHP, ASP, or JSP can be configured to use URL
schemes that encode script input parameters as a directory-like string following
the name of the script in the URL. Figure~\ref{rpo:fig:taint:parameter_simple}
shows a generic example where there is no parameter in the URL. Since the
crawler does not know the name of valid parameters, it simply appends the
payload as a subdirectory to the end of the URL. In this case, content injection
can occur if the page reflects the page URL or referrer into the response.

Different web frameworks handle path parameters slightly differently, which is
why we distinguish a few additional cases. If parameters are present in the URL,
we can distinguish these cases based on a number of regular expressions that we
generated. For example, parameters can be separated by slashes (see
Figure~\ref{rpo:fig:taint:parameter_php}) or semicolons (see
Figure~\ref{rpo:fig:taint:parameter_jsp}). When the crawler detects one of these
known schemes, it injects the payload into each parameter. Consequently, in
addition to URL and referrer reflection, injection can also be successful when
any of the parameters is reflected in the page.

\paragraph{Encoded Path}

This technique targets web servers such as IIS that decode encoded slashes in
the URL for directory traversal, whereas web browsers do not. Specifically, we
use \texttt{\%2F}, an encoded version of `\texttt{/}', to inject our payload
into the URL in such a way that the canonicalized URL is equal to the original
page URL (see Figure~\ref{rpo:fig:taint:path}). Injection using this technique
succeeds if the page reflects the page URL or referrer into its output.

\paragraph{Encoded Query}

Similar to the technique above, we replace the URL query delimiter `\texttt{?}'
with its encoded version \texttt{\%3F} so that web browsers do not interpret it
as such. In addition, we inject the payload into every value of the query
string, as can be seen in Figure~\ref{rpo:fig:taint:query}. CSS injection
happens if the page reflects either the URL, referrer, or any of the query
values in the HTML response.

\paragraph{Cookie}

Since stylesheets referenced by a relative path are located in the same origin
as the referencing page, its cookies are sent when requesting the stylesheet.
CSS injection may be possible if an attacker can create new cookies or tamper
with existing ones (a strong assumption compared to the other techniques), and
if the page reflects cookie values in the response. As shown in
Figure~\ref{rpo:fig:taint:cookie}, the URL is only modified by adding slashes to
cause path confusion. The payload is injected into each cookie value and sent by
the crawler as an HTTP header.

\subsubsection{Exploitability Detection}
\label{rpo:sec:methodology:exploitable}

Once a page has been found to be vulnerable to style injection using RPO, the
final step is to verify whether the reflected CSS in the response is evaluated
by a real browser. To do so, we built a crawler based on Google Chrome, and used
the Remote Debugging Protocol~\cite{debugging_protocol} to drive the browser and
record HTTP requests and responses. In addition, we developed a Chrome extension
to populate the cookie header in CSS stylesheet requests with our payload.

In order to detect exploitable pages, we crawled all the pages from the previous
section that had at least one reflection. Specifically, for each page we checked
which of the techniques in Figure~\ref{rpo:fig:taint_techniques} led to
reflection, and crafted the main URL with a CSS payload. The CSS payload used to
verify exploitability is different from the simple payload used to test
reflection. Specifically, the style directive is prefixed with a long sequence
of \texttt{\}} and \texttt{]} characters to close any preceding open curly
braces or brackets that may be located in the source code of the page, since
they might prevent the injected style directive from being parsed correctly. The
style directive uses a randomly-generated URL to load a background image for the
HTML body. We determine whether the injected style is evaluated by checking the
browser's network traffic for an outgoing HTTP request for the image.

\paragraph{Overriding Document Types}

Reflected CSS is not always interpreted by the browser. One possible explanation
is the use of a modern document type in the page, which does not cause the
browser to render the page in quirks mode. Under certain circumstances, Internet
Explorer allows a parent page to force the parsing mode of a framed page into
quirks mode~\cite{prssi}. To test how often this approach succeeds in practice,
we also crawled vulnerable pages with Internet Explorer~11 by framing them while
setting \texttt{X-UA-Compatible} to \texttt{IE=EmulateIE7} via a \texttt{meta}
tag in the attacker's page.
