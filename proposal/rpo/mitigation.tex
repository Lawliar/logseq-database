\subsection{Countermeasures}
\label{rpo:sec:mitigation}

Relative path overwrites rely on the web server and the web browser interpreting
URLs differently. HTML pages can use only absolute (or root-relative) URLs,
which removes the need for the web browser to expand relative paths.
Alternatively, when the HTML page contains a \texttt{<base>} tag, browsers are
expected to use the URL provided therein to expand relative paths instead of
interpreting the current document's URL. Both methods can remove ambiguities and
render RPO impossible if applied correctly.

Web developers can reduce the attack surface of their sites by eliminating any
injection sinks for strings that could be interpreted as a style directive.
However, doing so is challenging because in the attack presented in this paper,
style injection does not require a specific sink type and does not need the
ability of injecting markup. Injection can be accomplished with relatively
commonly used characters, that is, alphanumeric characters and
\texttt{()\{\}/"}.

Instead of preventing RPO and style injection vulnerabilities, the most
promising approach could be to avoid exploitation. In fact, declaring a modern
document type that causes the HTML document to be rendered in standards mode
makes the attack fail in all browsers except for Internet Explorer. Web
developers can harden their pages against the frame-override technique in
Internet Explorer by using commonly recommended HTTP headers:
\texttt{X-Content-Type-Options} to disable ``content type sniffing'' and always
use the MIME type sent by the server (which must be configured correctly),
\texttt{X-Frame-Options} to disallow loading the page in a frame, and
\texttt{X-UA-Compatible} to turn off Internet Explorer's compatibility view.
