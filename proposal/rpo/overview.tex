\subsection{Overview}

There are many existing techniques of how style directives could be injected
into a site~\cite{ccs2012scriptless,ccs2010cross_origin_css}. A relatively
recent class of attacks is Relative Path Overwrite (RPO), first proposed in a
blog post by Gareth Heyes~\cite{rpo} in 2014. These attacks exploit the semantic
disconnect between web browsers and web servers in interpreting relative paths
(\textit{path confusion}). More concretely, in certain settings an attacker can
manipulate a page's URL in such a way that the web server still returns the same
content as for the benign URL. However, using the manipulated URL as the base,
the web browser incorrectly expands relative paths of included resources, which
can lead to resources being loaded despite not being intended to be included by
the developer. We define a page as \textit{vulnerable} if:

\begin{itemize}

\item The page includes at least one stylesheet using a relative path.

\item The server is set up to serve the same page even if the URL is manipulated
by appending characters that browsers interpret as path separators.

\item The page reflects style directives injected into the URL or cookie. Note
that the reflection can occur in an arbitrary location within the page, and
markup or script injection are not necessary.

\item The page does not contain a \texttt{<base>} HTML tag before relative paths
that would let the browser know how to correctly expand them.

\end{itemize}

We define a vulnerable page as \textit{exploitable} if the injected style is
interpreted by the browser--that is, if an attacker can force browsers to render
the page in quirks mode. This can occur in two alternative ways:

\begin{itemize}

\item The vulnerable HTML page specifies a \textit{document type} that causes
the browser to use quirks mode instead of standards mode. The document type
indicates the HTML version and dialect used by the page;

\item Even if the page specifies a document type that would usually result in
standards mode being used, quirks mode parsing can often be enforced in Internet
Explorer~\cite{prssi}. Framed documents inherit the parsing mode from the parent
document, thus an attacker can create an attack page with an older document type
and load the vulnerable page into a frame. This trick only works in Internet
Explorer, however, and it may fail if the vulnerable page uses any anti-framing
technique, or if it specifies an explicit value for the \texttt{X-UA-Compatible}
HTTP header (or equivalent).

\end{itemize}

In this section, we present the first in-depth study of style injection
vulnerability using RPO. We extract pages using relative-path stylesheets from
the Common Crawl dataset~\cite{common_crawl}, automatically test if style
directives can be injected using RPO, and determine whether they are interpreted
by the browser.
