\subsection{Overview}
\label{norax:sec:overview}

Although creating isolations among mutually untrusted component with finer granularity can largely prevent in-process memory abuse, attackers could still exploit a vulnerable components directly. Numerous studies~\cite{boomerang,tsgx} have shown that attackers are able to exploit code running inside trusted executing environments, bypassing all deployed mitigations~\cite{controljujutsu,jitrop,rop}. 
Such attacks increasingly leverage code-reuse techniques~\cite{rop,ret2libc,jop} to gain control of vulnerable programs.
Since contemporary softwares widely employ code integrity protection techniques, such as data execution prevention (DEP~\cite{dep}), to prevent traditional code injection attacks. 
In code reuse attacks, a target application's control flow is manipulated in a way that snippets of existing code (called gadgets) are chained and run to carry out malicious activities.

Knowledge of process memory layout is a key prerequisite for code-reuse
attacks to succeed.  Attackers need to know the exact binary instruction locations in
memory to assemble the chain of gadgets. Commodity operating systems
widely adopt address space layout randomization (ASLR), which loads code binaries at
random memory locations unpredictable to attackers. Without knowing the locations of needed
code or gadgets, attackers cannot build code-reuse chains.  

However, \emph{memory
disclosure} attacks can use information leaks in programs to de-randomize code
locations, thus defeating ASLR.  Such attacks either
 read the program code (\emph{direct} de-randomization) or read code
pointers (\emph{indirect} de-randomization).  
Given that deployed ASLR techniques only randomize the load address of a large chunk of data or code, leaking a single code pointer or a small sequence
of code allows attackers to identify the corresponding chunk, infer its base address, and calculate the addresses of gadgets contained in the chunk. 

More sophisticated fine-grained ASLR
techniques~\cite{davi2013gadge,smashinggadget, ilr,aslp,binstir} aim
at shuffling code blocks within the same module to make it more difficult for attackers
to guess the location of binary instructions. Nevertheless, research by
Snow et al.~\cite{jitrop} proves that memory disclosure vulnerabilities can
bypass the most sophisticated ASLR techniques.

Therefore, a robust and effective defense against code-reuse attacks should combine fine-grained ASLR with memory disclosure prevention. 
Some recent works proposed to prevent memory disclosures using compile-time techniques~\cite{readactor,readactorpluplu,lr2}. Despite their effectiveness, these solutions cannot cover COTS binaries that cannot be easily recompiled and redeployed. These binaries constitute a significant portion of real-world applications that need protection. 


In this section, we present \NORAX~\footnote{\NORAX stands for \textbf{NO}
\textbf{R}ead \textbf{A}nd e\textbf{X}ecute.}, which protects COTS binaries from code memory disclosure attacks. 
The goal of \NORAX is to allow COTS binaries to take advantage of execute-only memory (XOM), a new security feature that recent AArch64 CPUs provide and is widely available on today's mobile devices. 
While useful for preventing memory disclosure-based code reuse~\cite{jitrop, brop}, XOM remains barely used by user and system binaries due to its requirement for recompilation. 
\NORAX removes this requirement by automatically patching COTS binaries and loading their code to XOM. As a result, when used together with ASLR, \NORAX enables robust mitigation against code reuse attacks for COTS binaries.
% It is worth noting that while we demonstrate \NORAX on Android, the ideas behind \NORAX are generally applicable to any AAarch64 platform. 

% \NORAX allows COTS binaries to be loaded in hardware-enforced XOM, a security feature
%supported by recent ARM CPUs (i.e., AArch64). Such CPUs are widely seen on today's mobile devices. 
%Although it is an effective defense against disclosure-based code reuse~\cite{jitrop, brop}, 
%Without \NORAX, to use the XOM feature, binaries need to be (re)built with the necessary compiler support. This requirement stands in between the valuable security feature and a large number of COTS binaries (e.g., all Android system executables and libraries) that are already running on AArch64 CPUs but were not compiled with XOM support. 
%XOM remains largely unused by apps and system libraries due to its requirement for recompiling programs into compatible binaries. 
%\NORAX removes this requirement. It automatically patches existing binaries and loads their code to XOM-enforced memory regions, without affecting binaries' normal execution. As a result, binaries without special (re)compilation can benefit from the hardware-backed XOM feature and be protected against code memory disclosure. Further, 

%It is worth noting that we use Android as the reference platform for building and evaluating \NORAX. However,  \NORAX's approach and techniques  are generally applicable to other AArch64  platforms. 

\NORAX consists of four major components: \emph{
\NDisassembler,
\NPatcher,
\NLoader,
and \NMonitor}. 
The first two perform offline binary analysis and transformation. They convert COTS binaries built for AArch64 without XOM support into one whose code can be protected by XOM during runtime. 
The other two components provide supports for loading and monitoring the patched, XOM-enabled binaries during runtime.  
The design of \NORAX tackles a fundamentally difficult problem: identifying data embedded in code segments, which are common in ARM binaries, and relocating such data elsewhere so that during runtime code memory pages can be made executable-only while allowing all embedded data to be readable. 

We apply \NORAX to Android system binaries running on
%commodity systems. Upon writing, NORAX has been fully implemented on the modern 
SAMSUNG Galaxy S6 and LG Nexus 5X devices. The results show that \NORAX on average slows down the transformed binaries by 1.18\% and increases their memory footprint by  2.21\%, suggesting \NORAX is practical for real-world adoption. 
