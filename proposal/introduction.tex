\section{Introduction}
\subsection{Low-level system software security}
% What is low-level system software
Low-level system software plays a pivot role in the whole cyberspace that we live in -- they run everywhere because they connect the users who send commands through them and the hardware that physically performs all kinds of computational tasks. 
Conceptually, they accept and parse commands from the user, break down the tasks, convey and execute them on the real hardware.  
They are essential for this information age and the world. 
% Some example of these low-level system software
Noticeable examples of these low-level system software for today are 1). the OS kernel (e.g., the Linux kernel), and 2). the firmware. The OS kernel runs mostly on desktop and server environments to efficiently drive the common commodity hardware pieces (e.g., hard drive, CPU and RAM) and deal with the users' requests while the firmware usually runs in a more constrained hardware environment like the Micro-control Unit (MCU) and is not as user-facing as the former. 

% Why their security is important
Because of such an important role they play, they are usually placed in a high-privileged execution environment and subsequently suffer the victim of extremely calculated and powerful cyberattacks~\cite{kuruvila_hardware-assisted_2021,maggi_attacks_2020,miller_remote_2015}. Due to the fundamental nature of these critical software, patching their vulnerabilities after deployment is usually delayed, inconvenient and error-prone which leaves a window for attacks.\todo{add citation here} As a result, detecting the potentially exploitable bugs before the software being deployed in practice is crucial to mitigate these security threats.  

Meanwhile, in terms of reasoning and analyzing software, symbolic execution as well as the closely-related fuzz testing have seen huge success in various aspects of securing normal user-space programs. 
In fact, seminal works \todo{add citation here} in these fields have identified countless dangerous bugs in various user-space programs. 
Thus, it would be very tempting to apply these extensively-studied bug detecting techniques to securing these low-level system software. 
% execution units sharing address space is great, flexible, enable high performant software designs.
%Modern operating systems manage running instances of software in the granularity of processes. 
%Each process is conceptually comprised of threads for parallel job processing. While threads have 
%independent copies of the basic data structure such as process context for OS to manage their execution, they by design share the same memory address space. Sharing address space has numerous benefits. In terms of performance, it allows low-overhead communication among the running units. For example, Threads can coordinate efficiently with locking mechanisms implemented based on variable shared in the address space (e.g, spinlock~\cite{spinlock}). In terms of development flexibility, sharing the same memory space allows developers to define global data structures accessible from different components of the software.
%These benefits outweigh the risk of security back in the early 60s when the design of virtual memory was introduced. It works well when most components in a software are developed by the same party. However, as the software engineering paradigm shifts, more parties are often involved in developing a single piece of software, and therefore, the lack of security boundaries among their components becomes a realistic concern and an opportunity for attacks. This trend is driven by the following factors.

% first problem, code entities are from different origins. but they are sharing the same address space.


However, unlike the normal user-space programs which enjoy a normalized and abstract execution environment provided by the operating system, low-level system software runs in much more complex and diverse environments. 
This inevitably poses unique technical challenges when applying these well-established bug detection techniques to them. 
We identify these technical challenges as follows: 

\point{Direct and intensive interaction with the hardware} 
Unlike user-space programs which interact with the hardware through the well-defined system calls, low-level system software directly interact with the hardware through hardware-provided interfaces, namely, 1). Memory-map Input and Output (MMIO), 2). Interrupt and 3). Direct Memory Access (DMA). 
So when designing a symbolic execution or fuzz testing framework, instead of focusing on handling the generic well-defined system call interfaces, one has to consider these hardware dependencies. 
This mainly incurs three problems: 
\begin{itemize}
    \item Limited visibility and computing resource: 
            User-space program enjoys much visibility and abundance of computing resources while the low-level system software is not meant for being debugged thus has much less visibility, debugging utility as well as computing resources (e.g, RAM). 
            This is problematic since most symbolic execution and fuzz testing frameworks are built on the assumptions that it has full visibility into the programs' states and relatively large amount of computing resources. 
    \item Hardware fidelity: 
            Due to the diversity of the hardware, apart from directly accessing them, many existing works tried to emulate the hardware or to abstract them away. 
            But hardware by itself is much like the software in terms of containing complex logic, thus simple abstraction usually lead to low hardware fidelity thus false positives. 
    \item Handle hardware interfaces properly: 
            Even with the aforementioned two problems solved, there is no existing work that tried to handle these three hardware interfaces like they do for the system calls. 
            It is still yet an open problem to handle the MMIO, interrupt, and DMA properly and authentically.  
\end{itemize}


\point{Large code base} 
Low-level system software tend to have relatively large code base mainly because 1). On the hardware's front, as the hardware usually contains complex functionalities, in order to fully drive each feature from the hardware, the software on top of it needs to enumerate and handle each one of them. 2). On the user's front, it needs to accept, parse and break down the relatively semantically-rich user commands and formulate them into machine-executable tasks, this also takes a large code base to achieve.

\point{Heavily event-driven design} 
Due to the critical role that the low-level system software is playing, its performance is carefully optimized and essential to all the applications that are built on top of it. 
As a result, an event-driven design is usually employed to further improve performance. 
However, rarely does the existing works that take these events into consideration during the design of their solutions. 
They usually just use a simple heuristic-based method, which often can not be realized in practice.



In this thesis, we design three different novel solutions targeting the aforementioned challenges respectively based on the current state-of-the-art. 

%\point{Conducting code-reuse attacks} This is a more generic form of executing code in
%the context of a victim process. The attack is mostly originated from a subverted vulnerable components, attacker then redirects execution to different code sections loaded in the same process. The goal is to impersonate as the compromised process for various purposes such as privilege escalation, sandbox-escaping, etc. Since the wide adoption of mitigation techniques like Data Execution Prevention (DEP~\cite{dep}), code-reuse attacks~\cite{ropnoreturn, ret2libc, ret2libc2, ret2libc3,jitrop} have become a common attacking technique.
\subsection{Thesis Statement}

In light of the current dire security situation in the low-level system software, also to apply symbolic execution as well as fuzz testing to better secure the low-level system software, 
my research takes three steps to refine and improve the symbolic execution and fuzz testing techniques to make them more fit for those critical low-level software that we daily depend on.

First, we need to have fast and high-fidelity access to the real hardware so that the quality of the software testing can be drastically improved. 
Current solutions either connect to the real hardware devices through debugging port or emulate them in the emulators. 
The former is costly, requires high customization, and does not support all the hardware features. 
The latter causes high false positives, i.e., the bugs it detected often cannot be realistically replayed in the real hardware.

Second, we need to tailor the costly symbolic execution towards the large code base that the low-level system software usually has. 
At least for certain type of bugs, we can optimize the symbolic execution strategy to reason those types of bugs more efficiently so that it can scale to a large code base. 


Lastly, we need to adapt those bug detection techniques towards a more event-driven design. The current bug detection techniques mainly focus on testing sequential code without worrying about interrupt, as interrupts are abstracted away by the operating system from the user-space programs. But interrupt serves as an essential part for low-level system software, thus we need to incorporate interrupts into the design for a more efficient and comprehensive testing. 

\subsection{Approaches Overview}
In this thesis, 
I present three novel techniques aiming to make symbolic execution and fuzz testing more efficient and effective for the critical low-level software. 

%first work: shreds aids developers to create use-to-use in-process memory isolation to protect their secret data
First, I present \textsc{SPEAR}\todo{add citation}, a new symbolic execution framework designed for embedded applications which uniquely provides high-fidelity and high-speed execution environment. 
Old symbolic execution designs suffer when the firmware interacts with the hardware. 
More specifically, due to their migrating firmware from its native environments to the emulators running on the workstation, they have to either emulate those missing peripherals or forward peripheral accesses to the real hardware. 
The former usually results in false program state, the latter is slow. 
\textsc{SPEAR} uniquely runs the concrete execution on the real hardware. 
The concrete execution is only responsible for reporting runtime information necessary for the expensive symbolic execution running on the workstation. 
This simple design ensures the highest of hardware fidelity since all concrete execution is carried out on the real hardware. 
It is also very fast out-performing the state-of-the-art symbolic execution framework by 1.5 to 2 times. 

%second work: norax provides xom primitive for combating information-leak based memory reuse attacks
Second, I present \textsc{KUBO}~\cite{kubo}, a scalable symbolic execution based static detector for undefined behaviors in the OS kernels.
Undefined behaviors plague the OS kernel because the OS kernel, as an important example of the system software, has to deal with all sort of different architectures and hardware. 
While C programming language, the predominant language used in the low level system software defines all different expected behaviors, it still has many behaviors that are left undefined due to the architectural differences. 
For example, signed integer overflow is a well-known undefined behavior in C language standard~\cite{misc:C-standard}. 
Although for one specific architecture, e.g., x86, its behavior is fixed, i.e., overflown integer will wrap around (i.e., \textbf{INT\_MAX + 1 = INT\_MIN}). 
However, since on other architectures (e.g., the GPU) overflown integer can saturate rather than wrapping around. 
These irreconcilable differences make it impossible for the C language standard to pick one expected behavior. 
Previous works cannot scale undefined behavior detection to the whole kernel code space. 
They usually just do an intra-procedural per-function analysis, leaving false positive to be as high as 87\%. 
In \textsc{KUBO}, we examined all the undefined-behavior-fixing patches in linux kernel over the past several years and systemized the bug triggering inputs from the userspace. 
In doing this, we constructed the bare minimal program slice for determining the triggerability of each UB bug thus scaling symbolic execution while taking advantage of its precision.
In our experiment we found 23 new undefined behavior bugs in the linux kernel out of which 14 are confirmed.

%Finally: savior, a verifiable bug-driven hybrid testing framework, to detect memory errors that could potentially lead to 
% in-process memory abuse attacks.
Finally, I propose \textsc{Shield},the first hybrid fuzz testing framework that takes firmware's event-driven design into consideration. 
Existing fuzz testing framework conceptually 
Differing from the existing hybrid testing tools, \textsc {Savior} prioritizes the concolic execution to solve branch predicates guarding more potential vulnerabilities and then drives fuzz testing to reach code with more potential vulnerabilities. 
Going beyond that, \textsc {Savior} inspects all vulnerable candidates along the running program path in concolic execution. 
By modeling the faulty situations with SMT constraints, it solves proofs of valid vulnerabilities and outputs concrete test cases. 
Our evaluation shows that the bug-driven approach outperforms the state-of-art code coverage driven hybrid testing tools in vulnerability detection. 
In our preliminary experiments comparing \savior with state-of-the-art software testing techniques on the widely used LAVA benchmark~\cite{lava}. 
Within 5 hours, in addition to triggering all bugs in {\tt base64}, {\tt md5sum} and {\tt uniq}, \savior found 1904 bugs in {\tt who} and many other unlisted bugs. This is, to the best of our knowledge, by far the best results in the existing literatures.

%On average, \textsc {Savior} detects vulnerabilities XX\% faster and discovers YY\% more 
%security violations. According to the experimental result on 14 well fuzzed benchmark programs, \textsc {Savior} triggers 2196 unique security violations within 24 hours.

