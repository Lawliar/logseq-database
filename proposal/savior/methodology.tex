\subsection{Methodology}
\label{savior:sec:methodology}

\savior's hybrid testing demands the static information of vulnerability 
labels and the domination relations. We gather the 
required information via static analysis on the LLVM IR of the target 
program. 

\point{Vulnerability Labeling} 
Currently, \savior leverages existing tools to provide vulnerability information. We have summarized that,
In general, any tool satisfying the following criteria can plug-and-play. 

\begin{itemize}
\item As \savior loads the labels prior to testing, the tool should 
be able to work statically on the code.

\item The tool should conduct sound analysis. In other words, 
we expect a complete set of labels which should mark all the potential 
vulnerabilities of interests. 
%
An unsound tool may miss a lot of true labels, which affects \savior 
in two ways. First, the weights of many non-covered branches are 
falsely lowered since the labels in their dominating code get reduced. 
This delays \savior to solve them and the defers the identification 
of vulnerabilities guarded by them. Second, \savior will lose guidance 
to those true labels and miss the opportunities to catch them 
during vulnerability verification. We understand that a such 
tool often produces many false postives. This may also 
reduce the effectiveness of \savior in prioritization. 
So we should employ additional analysis that can help reduce such false positives. 

\item An applicable tool has to summarize the triggering 
conditions once it labels a vulnerability.
The reason is that \savior relies on such conditions 
as guidance in vulnerability verification. Going beyond that, 
we only accept conditions that have data dependency with 
objects in the target program (\eg conditions that checks status of 
variables in the program). Otherwise our concolic execution will 
not correlate the program execution with the vulnerability conditions 
and hence, unable to verify the existence of bug. 
For instance, the widely used AddressSanitizer~\cite{serebryany2012addresssanitizer} 
verifies the vulnerability based on the status of its own red-zone,  
which are not applicable to \savior.

\item Engineering wise, we expect the tool to be compatible with LLVM IR, since 
\savior runs with LLVM IR. In addition, the tool needs to either marks the labels 
it plants or exports related information. This is mainly because \savior needs 
such information for prioritizing non-covered branches and 
vulnerability verification.  
\end{itemize}

In our current design, \savior employs 
LLVM UBSan~\cite{ubsanlist}, which conservatively capture 
a wide spectrum of undefined behaviors. 
By default, \savior only enables certain components in UBSan to 
labels issues that have explicit security consequences, 
including signed/unsigned integer overflow, floating cast overflow,
pointer arithmetic overflow and out-of-bounds array indexing. 
In addition, we patch the Clang front-end to attach UBSan labels
to the generated IR. 
%Note that UBSan natively instruments the vulnerability conditions, 
%which are associated with the internal variables. 
The vulnerability conditions created by UBSan mostly follow
the comprehensive summaries in~\cite{wang2013towards} 
and we omit their details. 

\point{Domination Analysis} This analysis proceeds with two phases 
which first construct the inter-procedure control flow graph (CFG) and then build the
pre-domination tree on the CFG. We present their specifics in the following.

To generate the inter-procedure CFG, we use an algorithm similar to the one 
implemented in SVF~\cite{sui2016svf}. The algorithm starts with building intra-procedure CFGs for 
individual functions and then connect them by the caller-callee relation.
To resolve indirect calls, it iteratively performs point-to analysis and expand 
the targets of indirect calls. In \savior, we utilize the Andersen's algorithm 
in point-to analysis for better efficiency. Note that our design is fully compatible 
with other point-to analyses that have higher precision 
(\eg flow-sensitive analysis~\cite{hardekopf2011flow}). 

Taking the above inter-procedure CFG as input, we run the algorithm proposed in~\cite{cooper2001simple}
to construct the domination tree. This enables us to understand the code regions dominated by each of 
the basic blocks and count the number vulnerability labels enclosed. 
For the convenience of access at the time of testing, we encode the dominated bug number 
(metadata in LLVM IR) at the beginning of each basic block. 
