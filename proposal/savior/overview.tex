\subsection{Overview}
\label{savior:sec:overview}

Facilitated by efforts from both the academia and the industry, software testing 
techniques gain remarkable advancement in finding security vulnerabilities.
In particular, people have deeply exploited fuzz testing~\cite{takanen2008fuzzing, 
zalewski2014american} and concolic execution~\cite{Sen07a,majumdar2007hybrid}
to disclose a great amount of vulnerabilities every year. Nevertheless, the 
evolution of fuzz testing and concolic execution is gradually impeded by their 
intrinsic limitations. On one hand, fuzz testing quickly covers
different code regions but hardly reaches compartments that are guarded by
complex conditions. On the other hand, concolic execution excels at solving conditions and
enumerating execution paths while it frequently sinks the exploration into code
segments that contain a large volume of execution paths (\eg loop).
Due to these limitations, using fuzz testing or concolic execution alone often
ends with a substantial amount of uncovered code within the given time frame. 
To enlarge code coverage, recent researches explore the idea of hybrid testing which 
assembles both fuzz testing and concolic execution~\cite{qsyminsu,driller,jfuzz}.

The insight of hybrid testing is to apply fuzz testing on the path exploration and 
leverage concolic execution to resolve complex conditions. A hybrid approach relies 
on fuzz testing to exhaustively explore code regions. Once fuzzing stops 
making progress, the hybrid approach switches to concolic execution which replays the 
seeds accumulated in fuzzing. While following the path pertaining to a given seed,
concolic execution examines at each of the conditional statements to check 
whether the other branch remains uncovered. If so, it
solves the constraints leading to the new branch. 
This enables concolic execution to generate a new seed which 
guides fuzz testing to deeper code space exploration.

As demonstrated in recent works~\cite{driller,qsyminsu}, hybrid testing creates new 
opportunities towards high code coverage. However, it still adopts the code-driven philosophy, 
which unfortunately falls short of the needs of vulnerability mining.
First, hybrid testing treats all the branches indiscriminately. However, 
most of these branches might lead to code lack of vulnerabilities and the exploration on 
them involves expensive computation (\eg constraint solving and extra fuzzing). 
Consequently, hybrid testing often exhausts the assigned CPU cycles way before it could 
resolve the branches guarding vulnerabilities. 
%
Second, hybrid testing can fail to catch a vulnerability even if it reaches the vulnerable 
code region following a correct path. This is mainly because hybrid testing exercises new branch in the manner of random execution, which oftentimes has low chance to satisfy the subtle conditions to trigger the vulnerability. 

In this section, I propose \savior (abbreviation for {\bf
S}peedy-{\bf A}utomatic-{\bf V}ulnerability-{\bf I}ncentivized-{\bf OR}acle),
a hybrid testing tool that evolves towards being bug-driven and verification-based. To fulfill 
this goal, \savior incorporates two novel techniques as follows. 
\point {Bug-driven Prioritization} Instead of solving the unreached branches 
in an indiscriminate manner, \savior prioritizes those that have higher 
promise in leading to vulnerable code. Specifically, prior to the testing, 
\savior labels all the potential sites with vulnerability in the target program. As 
there lacks theories to obtain the ground truth of vulnerabilities, \savior follows 
the extant schemes~\cite{ubsanlist, Dietz:2012} to conservatively capture all the possibilities. 
In addition, \savior statically computes the code regions dominated by each branch. 
At the time of testing, \savior prioritizes concolic execution on branches that dominate 
code with more vulnerability labels. Since such branches guard higher volumes 
of vulnerabilities, advancing the schedule to reach them can significantly 
expedite the discovery of vulnerabilities. 

\point {Bug-guided Search} 
Going beyond accelerating vulnerability finding, \savior also comprehensively seeks 
for vulnerabilities along the paths explored by fuzzing. 
While \savior executes a seed with concolic execution, it revisits each 
of the vulnerability label that remain unconfirmed on the execution path.
That is, \savior constructs the constraints to trigger such a vulnerability following the current path. 
If the synthesized constraints are satisfiable, \savior generates a test 
case as a proof of this vulnerability. Otherwise \savior verifies that the vulnerability 
is un-triggerable by arbitrary inputs going through this path. 

