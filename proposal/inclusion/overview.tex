\subsection{Overview}
\label{inclusion:sec:overview}

While the same origin policy (SOP) enforces a modicum of origin-based separation
between code and data from different principals, developers have clamored for
more flexible sharing models provided by, e.g., Content Security Policy
(CSP)~\cite{csp_spec}, Cross-Origin Resource Sharing (CORS)~\cite{cors_spec},
and postMessage-based cross-frame communication. These newer standards permit
greater flexibility in performing cross-origin inclusions, and each come with
associated mechanisms for restricting communication to trusted origins. However,
recent work has shown that these standards are difficult to apply securely in
practice~\cite{ndss2013postman,raid2014csp}, and do not necessarily address the
challenges of trusting remote inclusions on the dynamic Web. In addition to the
inapplicability of some approaches such as CSP, third parties can leverage their
power to bypass these security mechanisms. For example, ISPs and browser
extensions are able to tamper with HTTP traffic to modify or remove CSP rules in
HTTP responses~\cite{usenixsec2015webeval,sp2015adinjection}.

In this section, we propose an in-browser approach called \excision to
automatically detect and block malicious third-party content inclusions as web
pages are loaded into the user's browser or during the execution of browser
extensions. Our approach does not rely on examination of the content of the
resources; rather, it relies on analyzing the sequence of inclusions that leads
to the resolution and loading of a terminal remote resource. Unlike prior
work~\cite{ccs2012madtracer}, \excision resolves \emph{inclusion sequences
(chains)} through instrumentation of the browser itself, an approach that
provides a high-fidelity view of the third-party inclusion process as well as
the ability to interdict content loading in real-time. This precise view also
renders ineffective common obfuscation techniques used by attackers to evade
detection. Obfuscation causes the detection rate of these approaches to degrade
significantly since obfuscated third-party inclusions cannot be traced using
existing techniques~\cite{ccs2012madtracer}. Furthermore, the in-browser
property of our system allows users to browse websites with a higher confidence
since malicious third-party content is prevented from being included while the
web page is loading.

We implemented \excision as a set of modifications to the Chromium browser, and
evaluated its effectiveness by analyzing the Alexa Top 200K over a period of 11
months. Our evaluation demonstrates that \excision achieves a 93.39\% detection
rate, a false positive rate of 0.59\%, and low performance overhead. We also
performed a usability test of our research prototype, which shows that \excision
does not detract from the user's browsing experience while automatically
protecting the user from the vast majority of malicious content on the Web. The
detection results suggest that \excision could be used as a complementary system
to other techniques such as CSP.
