\subsection{Implementation}
\label{adinjection:sec:impl}

We implemented \origintracer as a set of modifications to the Chromium browser.
In particular, we modified both Blink and the extension engine to track the
provenance of content insertion, modification, and removal according to the
semantics presented in Section~\ref{adinjection:sec:design}. These modifications
also implement provenance indicators for suspicious content that does not
originate from the publisher.

\subsubsection{Tracking Publisher Provenance}

A core component of \origintracer is responsible for introducing and propagating
provenance label sets for DOM elements. To track the modifications, we monitor
JavaScript execution in Blink. In order to have a complete mediation of all DOM
modifications to Web page, \texttt{Node} class in Blink implementation was
instrumented in order to assign provenance label sets for newly created or
modified elements using the label set applied to the currently executing script.

\subsubsection{Tracking Extension Provenance}
\label{adinjection:sec:impl:extension}

Chromium extensions can manipulate the web page's content by injecting
\emph{content scripts} into the web page or using the \texttt{webRequest} API.
Content scripts are JavaScript programs that can manipulate the web page using
the shared DOM, communicate with external servers via \texttt{XMLHttpRequest},
invoke a limited set of \texttt{chrome.*} APIs, and interact with their owning
extension's background page. By using \texttt{webRequest}, extensions are also
able to modify and block HTTP requests and responses in order to change the web
page's DOM. In this work, we only track content modifications by content scripts
and leave identifying ad injection by \texttt{webRequest} for future engineering
work.

\subsubsection{Content Provenance Indicators}
\label{adinjection:sec:impl:indicators}

Given DOM provenance information, \origintracer must first
\begin{inparaenum}[\itshape i)\upshape]
    \item identify when suspicious content modifications -- e.g.,
    extension-based ad injection -- has occurred, and additionally
    \item communicate this information to the user in an easily comprehensible
    manner.
\end{inparaenum}

To implement the first requirement, our prototype monitors for content
modifications where a subtree of elements are annotated with label sets that
contains a particular extension's label. There are several possible options to
communicate content provenance. In our current prototype, provenance is
indicated using a configurable border color of the root element of the
suspicious DOM subtree. This border should be chosen to be visually distinct
from the existing color palette of the web page. Finally, a tooltip indicating
the root label is displayed when the user hovers their mouse over the DOM
subtree.
