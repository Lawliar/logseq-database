\subsection{Overview}
\label{adinjection:sec:intro}

While ad injection cannot necessarily be categorized as an outright malicious
activity on its own, it is highly likely that many users in fact \emph{do not
want or expect} browser extensions to inject advertisements or other content
into Web pages. Moreover, it can have a significant impact on the security and
privacy of both users as well as website publishers. For example, recent studies
have shown that ad-injecting extensions not only serve ads from ad networks
other than the ones with which the website publishers intended, but they also
attempt to trick users into installing malware by inserting rogue elements into
the web page~\cite{sp2015adinjection,www2015adinjection}.

To address this problem, several automatic approaches have been proposed to
detect malicious behaviors (e.g., ad injection) in browser
extensions~\cite{www2015adinjection,usenixsec2014hulk,usenixsec2015webeval}. In
addition, centralized distribution points such as Chrome Web Store and Mozilla
Add-ons are using semi-automated techniques for review of extension behavior to
detect misbehaving extensions. However, there is no guarantee that analyzing the
extensions for a limited period of time leads to revealing the ad injection
behaviors.

Although ad injection can therefore potentially pose significant risks, this
issue is not as clear-cut as it might first seem. Some users might legitimately
want the third-party content injected by the extensions they install, even
including injected advertisements. This creates a fundamental dilemma for
automated techniques that aim to identify clearly malicious or unwanted content
injection, since such techniques cannot intuit user intent and desires in a
fully automatic way.

To resolve this dilemma, we present \origintracer, an in-browser approach to
highlight extension-based content modification of web pages. \origintracer
monitors the execution of browser extensions to detect content modifications
such as the injection of advertisements. Content modifications are visually
highlighted in the context of the web page in order to
\begin{inparaenum}[\itshape i)\upshape]
     \item notify users of the presence of modified content, and
     \item inform users of the \emph{source} of the modifications.
\end{inparaenum}
With this information, users can then make an informed decision as to whether
they actually want these content modifications from specific extensions, or
whether they would rather uninstall the extensions that violate their
expectations.

\origintracer assists users in detecting content injection by distinguishing
injected or modified DOM elements from genuine page elements. This is performed
by annotating web page DOM elements with a \emph{provenance label set} that
indicates the principal(s) responsible for adding or modifying that element,
both while the page is loading from the publisher as well as during normal
script and extension execution. These annotations serve as trustworthy,
fine-grained provenance indicators for web page content. \origintracer can be
easily integrated into any browser in order to inform users of extension-based
content modification. Since \origintracer identifies all types of content
injections, it is able to highlight all injected advertisements regardless of
their types (e.g., flash ads, banner ads, and text ads).

We implemented a prototype of \origintracer as a set of modifications to the
Chromium browser, and evaluated its effectiveness by conducting a user study.
The user study reveals that \origintracer produced a significantly greater
awareness of third-party content modification, and did not detract from the
users' browsing experience. Our results also suggests that \origintracer can be
used as a complementary system to ad blocking systems such as
AdblockPlus~\cite{adblockplus} and Ghostery~\cite{ghostery}.
