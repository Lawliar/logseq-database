\section{Reading Papers}
\label{sec:read}
Reading is another gigantic(I intend to use these kind of "scary/exaggerated" words) part of PhD's life.
\subsection{Amount}
\subsection{Why this sentence is written.}
%% Wow, this sentence is long
Instead of passively(often unwillingly) scanning through the content of each paper (whose output for me is usually just the 
sense of minor achievement of being able to put the paper from "to-read" list to "have read" list without enough substantial knowledge gain(both technically and literarily)),
I found it beneficial to ask why the author wrote this sentence(paragraph)? 
Often times, this question can be answered even uncontiously(I think this is an ideal state of paper reading).

Many people said, "read paper like you're the reviewer". Undoubtedly there is some truth to it, as reviewing the paper forces you to grind the sentences and 
evetually get the most out of it.
But I often found this burdensome, mainly because reviewership %is this even a word?
means huge responsibilities(locating the logical errors, justifying the fulfillment of the evaluation e.t.c.).
It takes me huge amout of attention and energy to fully understand the writing and technical chain behind one paper.
Maybe for senior members, this can be done easily, but for a newbie paper reader as me, 
this is more challenging than what I can handle just with the energy I can generate for say, a single day.

So, I see the approach mentioned above a less-demanding version of this reviewership approach.
This doesn't need you to be an expert as you're required to be if you were to review this paper,
it's just based on the writing flow and technical facts present in the paper.