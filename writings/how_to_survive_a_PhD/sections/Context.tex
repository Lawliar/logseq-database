\section{Why these exist?}
An enormous part of a PhD's life is undoubtedly to learn different techniques dedicated for an ultimate problem.
For most, especially for beginners, a puzzling thing is the huge amount of existing practices that you might need to get familiar with.
If you just glimpse through them, you might be frustrated by the subtle differences among them.
However, if you decided to figure them out, the progress of your own project(dedicated to get something publised) can be severely impacted.

If you talked to some senior members who should have been through all this, a typical answer you can get is that, ""
Different tools and techniques are countless, it is easy to be overwhelmed by the huge number of tools that aimed at the same goal.
This problem is worsen in acedemia where there are many prototypes projects whose contribution is unknown until you read certain lines of its paper.

As a PhD students, it's your duty to at least get familiar with these different technologies but sometimes, you find yourself burried by some 
technical details, or you find something too easy to grasp while overlooking its importance.

