\section{Why these exist?}
An enormous part of a PhD's life is undoubtedly to learn different techniques dedicated for an ultimate problem.
For most, especially for beginners, a puzzling thing is the huge amount of existing practices that you might need to get familiar with.
If you just glimpse through them, you might be frustrated by their subtle differences.
However, if you decided to figure them out, the progress of your own project(dedicated to get something publised) can be severely impacted.

If you talked to some senior members who should have been through all this, a typical answer you can get is that, ""
Different tools and techniques are countless, it is easy to be overwhelmed by the huge number of tools that aimed at the same goal.
This problem is worsen in acedemia where there are many prototypes projects whose contribution is unknown until you read certain lines of their papers.

Just like in the early period of my trying to survey as many papers as possible, I was constantly frustrated by the repeated attempts of these papers, 
asking "if it's like what it's said in the paper, then how come we still have this problem?".
For this, I found it helpful to add a little historical background to help me understand the importance of these paper.

