\section{What's magic and what's not}
\label{sec:magic}
For most, even for the CS-backgrouned students, a balance between being ama(u)zed by the "magic" of computers and figuring out everything behind the magic 
has accompanied the learning process especially in the beginning period.

We always love the "magic", the magic can be "download CodeBlocks, create a C file as instructed, and write Hello world, and press "compile and run"", and boom, you make it!
There is no denying the fact that, the magic is why computers are so facinating to the public, the pioneers tried very hard to abstract things to make computers comprehensible even for the laymans. 
What make matters worse(better), the magic are all around the learning process, the existance of magic seems normal for many fields mainly because it freed us to do more high level staff with much shorter time.

It's worth noting that, "magic" is also often used to decribe sth works perfectly well but nobody could really explain why it works that way,
e.g., the "magic" of neural network whose "magical success" is still open to many interpretations and researches.
As in contrast, the "magic" I mean here is more like a highly-developed mature research field.
(It is said that[need citation], every research field, from a rather long-term perspective, can be divided into 4 phases 1. make it work, 2. make it efficient 3. make it invisible 4. make it a component of a greater solution).
The "magic" here are like the research problems that already developed into the 4th phase i.e., no stone is left unturned for this problem, everything is well-considered.
So much so that the contemporary works take them as granted. 

I'm not sure for other computer fields, for a research field as messy as system security, I struggled a lot to figure out what should I take as magic, and what should not.
And I frequently encountered the following magic which I think I should not take for granted: 
\subsection{ELF formats and linkers}