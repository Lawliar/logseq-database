\section{Time management}
\label{sec:time}
During the process of trying out different time management strategies, although I still think I'm in the middle of the tunnel, I definitely see some lights ahead.
More specifically, I tried 2 completely different strategies in the past year, and at least I think conclusion can be drawn for comparing these two.

The first one is quite simple which is just to "devote anytime you can have to a project".
I see this as an extention from my past experience which is "I always succeed when I've decided to achieve something".
For example, when I'm determined to have good grades, I can achieve high self-discipline and work tirelessly for weeks for the final week, and it worked everytime.
Even for the first research project I was assigned to during my master's degree, I work closely with my mentor and implemented a prototype 
which my senior peer has spent more than a year on and could not deliver.

So, although not completely conciously aware of this, I always attribute the futility of my peers to their inactivity or lazyness.

But as I was trying to implement the same strategy to my first "serious" research project for my PhD, I struggled a lot especially in the beginning perioud.
The magic seems gone.
\subsection{Learning to relax}
https://mp.weixin.qq.com/s/t1otw17tCwgvvx4lWoQiLA
1. Relaxing is important
2. Doesn't have to sth fancy(e.g. going out for movie, vacation), as long as sth makes you feel relaxed(for me it's sometimes hard to find).