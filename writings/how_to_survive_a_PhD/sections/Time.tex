\section{Time management}
\label{sec:time}
During the process of trying out different time management strategies, although I still think I'm in the middle of the tunnel, I definitely see some lights ahead.
More specifically, I tried 2 completely different strategies in the past year, and at least I think conclusion can be drawn for comparing these two.

The first one is quite simple which is just to "devote anytime you can have to a project".
I see this as an extention from my past experience which is "I always succeed when I've decided to achieve something".
For example, when I'm determined to have good grades, I can achieve high self-discipline and work tirelessly for weeks for the final week, and it worked everytime.
Even for the first research project I was assigned to during my master's degree, I work closely with my mentor and implemented a prototype 
which my senior peer has spent more than a year on and could not deliver.

So, although not completely conciously aware of this, I always attribute the futility of my peers to their inactivity or lazyness.

But as I was trying to implement the same strategy to my first "serious" research project for my PhD, I struggled a lot especially in the beginning perioud.
The magic seems gone.
\subsection{Learning to relax}
Now looking back at those times, I figured, one major reason for the disappearance of that power is that, relaxing was tiring.
Mainly due to the complete change of environment, I have to construct a whole chain of life i.e., use what trick to concentrate and use what to relax myself.
And since my focus has been on the research for the whole time, I didn't pay much attention to these trivia and just picked whatever comes handy.
The main tool I used to relax myself during that time ws anime and youtube, for one I didn't really enjoy those anime and videos
(I just thought now I should relax, so these come handy), plus my English was not good enough to digest the content effortlessly 
like my Chinese(my native language), and I tried to really exercise my English with that.

As a consequence, those "relaxing" activities turned out wearing me out. A good example is, everytime after I spent my "leisure time", I could never get enough and felt tired.
This severely damaged my motives and energy for the serious work i.e., research.

https://mp.weixin.qq.com/s/t1otw17tCwgvvx4lWoQiLA
1. Relaxing is important
2. Doesn't have to sth fancy(e.g. going out for movie, vacation), as long as sth makes you feel relaxed(for me it's sometimes hard to find).
\subsection{Plan your schedule like a pro}
How to balance your work and life no doubt comes first to your time management skill. 
If finding the activity that can relax yourself the most is the ultimate goal for the "life" part, 
some strategies are also needed for the "work" part.

Unlike "life" part whose quality largely depends on your subjective feelings i.e., if you feel truely relaxed or not, the quality of the "work" part is usually
meassured by some relatively objective metrics like productivity, the amout of tasks you get finished in the given amout of time etc,.
Despite the fact that the concept of "productivity" is rather objective, how to be productive can be really mythical. 
As a consequence, in the past year, I usually just mess around waiting for moment of concentration to come, and the result is usually undesirable.

So, just like you can find certain activities to make yourself relaxed(if you're lucky enough), you can also bridge the time from the moment when you should work
to the moment you turn productive. 

Idea-wise, I choose to kick off with some light-hearted learning of some classic well-studied fields that you feel interested, 
for example, I found "ELF format and linker" is sth that always comes into sight when I'm reading papers but I never truly know how it works
i.e., those things that look "magic"~\ref{sec:magic}. These things are well-studied i.e., no stone is left unturned, 
so you can just admire its beaty and learn from how it's designed.
(I cannot remember exactly, but it's said that there are 4 phrases for the development of a research field: to make it possible, to make it work fast,..., to make it invisible),
the thing that I choose here is those "invisible" things, or the "magic" like I describe above.